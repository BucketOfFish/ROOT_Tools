\section{Datasets}
\label{sec:datasamples}

%%%%%%%%%%%%%%%%%%%%%%%%%%%%%%%%%%%%%%%%%%%%%%%%%%%%%%%%%%%%%%%%%%%%%%%%%%%%%

\subsection{Data samples}
\label{sec:data}

This note analyzes the complete dataset collected with the ATLAS detector 
at $\sqrt{s}=8\TeV$.
%, corresponding to a total integrated luminosity of
% $\int\!{\cal L}\,{\rm d}t\simeq 20.3~\ifb$. 
%Further details on requirements on data quality and GRL can be found 
%in~\cite{COMB_INT}.
To assure good data quality, a good run list (GRL) selection is applied.
This GRL is taken from the official Data Preparation Group 
webpage~\cite{grl}, using version \texttt{v61\_pro14-02\_DQDefects-00-01-00\_PHYS \_StandardGRL\_All\_Good}. 
Since in the analysis electrons, muons, jets as well as \met\ are 
required, it is mandatory that basically all sub--systems of the 
ATLAS detector are in good condition. This is reflected in the specific
GRL, which assures good data quality for all relevant sub--systems. 
Applying this good run list results in a loss of approx. 6\% of data, leaving an integrated luminosity of $\int \mathscr{L}\mathrm{dt} = 20.3\,\ifb$ for analysis.
For the 2012 dataset the preliminary luminosity uncertainty is $\pm$ 2.8 \% based on the calibration procedure described in~\cite{lumi}.
%Data streams and trigger used for each analysis are briefly summarized in Table~\ref{tab:TrigStream}. 
The details of the trigger strategy are described in Section~\ref{sec:trigger}.


 
%The requirement that all sub-systems of the ATLAS experiment are working 
%acceptably reduces the total integrated luminosity to
% $\int_\text{GRL}\!{\cal L}\,{\rm d}t=(20.3\pm0.6)~\ifb$ with a 
% luminosity measurement uncertainty equal to 
% $2.8\%$~\cite{Aad:2011dr} {\bf TO BE CHECKED}. 
% The subdetector and data quality criteria are imposed using a 
% good-run list (GRL) taken from the official Data Preparation Group
%  webpage, version 
%  \\ {\tt \footnotesize data12\_8TeV.periodAllYear\_DetStatus-v58-pro14-01\_DQDefects-00-00-33\_PHYS\_StandardGRL\_All\_Good.xml}, \\ 
%  covering the entire 2012 proton-proton run period. 
%Data collected by both the {\tt egamma} and {\tt muon} streams are used.

%Events are selected using dilepton triggers for which the offline transverse
% momentum (\pT) threshold is the plateau of the turn-on curve. 
% For the dielectron selection, each event must fire either the 
% {\tt EF\_e24vh\_medium1\_e7\_medium1} or the {\tt EF\_2e12Tvh\_loose1} trigger.
%  Dimuon events are selected if they fire the {\tt EF\_mu18\_tight\_mu8\_EFFS} or
%   the {\tt EF\_2mu13} trigger. In addition, $e\mu$ events, used in the estimation
%    of some SM backgrounds, are selected if they pass the 
%    {\tt EF\_e12Tvh\_medium1\_mu8} or the {\tt EF\_mu18\_tight\_e7\_medium1} trigger. 

%%%%%%%%%%%%%%%%%%%%%%%%%%%%%%%%%%%%%%%%%%%%%%%%%%%%%%%%%%%%%%%%%%%%%%%%%%%%%

\subsection{Background Monte Carlo}
\label{sec:BackgroundMC}

The main SM backgrounds considered in this analysis are the \dyjets, $\ttbar$, single-top, 
and diboson processes. The evaluation of the instrumental \dyjets~background, 
which has \met~that is mostly due to hadronic mismeasurement, is made using 
a data-driven method. The $WW$, $\ttbar$, $Wt$, $\tau\tau$ backgrounds are 
estimated using $e\mu$ events. Minor backgrounds, such as dibosons, are estimated 
using Monte Carlo (MC) simulation, see section~\ref{sec:background}. 
%The QCD and inclusive \wjets~backgrounds are negligible. (CHECK)
%An overview of the main background samples is also given 
%in~\cite{COMB_INT}.

All Monte Carlo samples used in the analysis are from the official MC12a Monte Carlo production. 
These samples use the production tag \texttt{p1328}. 
Tables~\ref{tab:mcsample_ttbar} - \ref{tab:mcsample_dy}
%tab:mcsample_wjet, tab:mcsample_zjet, tab:mcsample_dy, tab:mcsamle_singletop, tab:mcsample_ttbarV, tab:mcsample_diboson, 
document all the simulated samples used in this note with their respective
cross-sections.
Minimum bias events generated with Pythia8 are overlaid in the digitization phase,
where average number of interactions per event $\langle \mu \rangle$ ranges between 0 to 40.
In order to reproduce the pile-up conditions of the data, the $\langle \mu \rangle$ distribution is 
reweighted according to the one in data using {\tt PileupReweighting} package.
Due to the fact that the total inelastic cross-section is not known completely, the number of vertices $N_{vtx}$ distribution 
is not reproduced even after the $\langle \mu \rangle$ reweighting.
Since what has much effect on the analysis is $N_{vtx}$, we rescale the $\langle \mu \rangle$ by $1/1.11$ before calculating 
the reweighting factor, which results in a better agreement of $N_{vtx}$ between data and Monte Carlo.
%To address the impact of those uncertainties for this analysis, 
%the $\langle \mu \rangle$ distribution is scaled by the factors 0.9 and 1.1 around the nominal value, 
%the resulting changes in MC event counts are then included in the final results. 

The \dyjets\ background is notoriously difficult to estimate in high \met\ and \HT\ regions. 
In order to check event level distributions in the signal region extra $Z+\text{jets}$ MC was requested in June 2014. 
To increase the signal region statistics an increase in the statistics of Dataset Numbers 167821-167826 and 167833-167838 was requested. 
These samples are indicated in Table~\ref{tab:mcsample_zjet}. 
For the lower \pt\ $Z$ samples very few events survive the harsh \met\ and \HT\ cuts of the SR. 
To ensure reasonable statistics from these samples in this region an \HT-filter was applied to the samples. 
These samples are summarised in Table~\ref{tab:mcsample_HTFilter}. 
More details on the studies relating to these extra samples can be found in Appendix~\ref{HTfilter}.

%%%%%%%%%%%%%%%%%%%%%%%%%%%%%%%%%%%%%%%%%%%%%%%%%%%%%%%%%%%%%%%%%%%%%%
%  ttbar
%%%%%%%%%%%%%%%%%%%%%%%%%%%%%%%%%%%%%%%%%%%%%%%%%%%%%%%%%%%%%%%%%%%%%%%%
\begin{table}[htb]
\begin{center}
%\resizebox{\textwidth}{!}{
\begin{tabular}{|p{1.7cm}|p{5cm}|p{4.5cm}|p{3cm}|}
\hline
Sample ID 	& Name 					& Generator 		& $\sigma$ [pb]\\
\hline\hline
117050	& ttbar &{\sc Powheg+Pythia}	&	253 $\times$ 0.543 \\
\hline
\end{tabular}
%}
\caption{Summary table of the \ttbar\ Monte Carlo samples used in the analysis. $\sigma$ is a NLO+NLL
cross-section times the fraction of none hadronic top decay.}
\label{tab:mcsample_ttbar}
\end{center}
\end{table}

%%%%%%%%%%%%%%%%%%%%%%%%%%%%%%%%%%%%%%%%%%%%%%%%%%%%%%%%%%%%%%%%%%%%%%%%
% single-top
%%%%%%%%%%%%%%%%%%%%%%%%%%%%%%%%%%%%%%%%%%%%%%%%%%%%%%%%%%%%%%%%%%%%%%%%
\begin{table}[htb]
\begin{center}
%\resizebox{\textwidth}{!}{
%\begin{tabular}{|c|c|c|c|}
\begin{tabular}{|p{1.7cm}|p{5cm}|p{4.5cm}|p{3cm}|}
\hline
Sample ID 	& Name 					& Generator 		& $\sigma$ [pb]\\
\hline\hline
%110101 & singletop\_tchan\_l	& {\sc AcerMC} & 25.8 $\times$ 1.10 \\ 
%110119 & st\_schan\_lep              	& {\sc Powheg} & 1.64 $\times$ 1.11 \\
%110140 & st\_Wtchan\_incl\_DR		& {\sc Powheg} & 20.5 $\times$ 1.09 \\
110141 & st\_Wtchan\_dilepton\_DR	& {\sc Powheg+Pythia} & 2.15 $\times$ 1.09 \\ 
179991 & tZ\_Wtchan\_Zll                        & {\sc MadGraph+Pythia} & 0.0041 \\
179992 & tZ\_stchan\_Zll                        & {\sc MadGraph+Pythia} & 0.0312 \\
\hline
\end{tabular}
%}
\caption{Summary table of the single top Monte Carlo samples used in the analysis. $\sigma$ is a LO cross-section times $k$-factor where applicable.}
\label{tab:mcsample_singletop}
\end{center}
\end{table}
%%%%%%%%%%%%%%%%%%%%%%%%%%%%%%%%%%%%%%%%%%%%%%%%%%%%%%%%%%%%%%%%%%%%%%%%
% ttbar+V
%%%%%%%%%%%%%%%%%%%%%%%%%%%%%%%%%%%%%%%%%%%%%%%%%%%%%%%%%%%%%%%%%%%%%%%%
\begin{table}[htb]
\begin{center}
%\resizebox{\textwidth}{!}{
%\begin{tabular}{|c|c|c|c|}
\begin{tabular}{|p{1.7cm}|p{5cm}|p{4.5cm}|p{3cm}|}
\hline
Sample ID 	& Name 					& Generator 		& $\sigma$ [pb]\\
\hline\hline
119353 & ttbarW		& {\sc MadGraph+Pythia} & 0.104   $\times$ 1.17 \\
174830 & ttbarWjExcl	& {\sc MadGraph+Pythia} & 0.0534  $\times$ 1.17 \\
174831 & ttbarWjjIncl	& {\sc MadGraph+Pythia} & 0.0415  $\times$ 1.17 \\
119355 & ttbarZ		& {\sc MadGraph+Pythia} & 0.0677  $\times$ 1.35 \\
174832 & ttbarZjExcl	& {\sc MadGraph+Pythia} & 0.0454  $\times$ 1.35 \\
174833 & ttbarZjjIncl	& {\sc MadGraph+Pythia} & 0.0398  $\times$ 1.35 \\
119583 & ttbarWW	& {\sc MadGraph+Pythia} & 0.00092 \\
\hline
\end{tabular}
%}
\caption{Summary table of the \ttbar\ plus W/Z Monte Carlo samples used in the analysis. $\sigma$ is a LO cross-section times $k$-factor where applicable.}
\label{tab:mcsample_ttbarV}
\end{center}
\end{table}
%%%%%%%%%%%%%%%%%%%%%%%%%%%%%%%%%%%%%%%%%%%%%%%%%%%%%%%%%%%%%%%%%%%%%%%%
% Dibosons (Powheg)
%%%%%%%%%%%%%%%%%%%%%%%%%%%%%%%%%%%%%%%%%%%%%%%%%%%%%%%%%%%%%%%%%%%%%%%%
\begin{table}[htb]
\begin{center}
%\resizebox{\textwidth}{!}{
%\begin{tabular}{|c|c|c|c|}
\begin{tabular}{|l|l|l|l|}
\hline
Sample ID 	& Name 					& Generator 		& $\sigma$ [pb]\\
\hline\hline
126928	& WpWm\_ee					& {\sc Powheg+Pythia8} & 0.598 $\times$ 1.08 $\times$ 1 \\
126929	& WpWm\_me					& {\sc Powheg+Pythia8} & 0.597 $\times$ 1.08 $\times$ 1 \\
126930	& WpWm\_te					& {\sc Powheg+Pythia8} & 0.598 $\times$ 1.08 $\times$ 1 \\
126931	& WpWm\_em					& {\sc Powheg+Pythia8} & 0.598 $\times$ 1.08 $\times$ 1 \\
126932	& WpWm\_mm					& {\sc Powheg+Pythia8} & 0.597 $\times$ 1.08 $\times$ 1 \\
126933	& WpWm\_tm					& {\sc Powheg+Pythia8} & 0.597 $\times$ 1.08 $\times$ 1 \\
126934	& WpWm\_et					& {\sc Powheg+Pythia8} & 0.597 $\times$ 1.08 $\times$ 1 \\
126935	& WpWm\_mt					& {\sc Powheg+Pythia8} & 0.598 $\times$ 1.08 $\times$ 1 \\
126936	& WpWm\_tt					& {\sc Powheg+Pythia8} & 0.598 $\times$ 1.08 $\times$ 1 \\
126937	& ZZ\_4e\_mll4\_2pt5				& {\sc Powheg+Pythia8} & 0.0769 $\times$ 1 $\times$ 0.908 \\
126938	& ZZ\_2e2mu\_mll4\_2pt5			        & {\sc Powheg+Pythia8} & 0.176  $\times$ 1 $\times$ 0.827 \\
126939	& ZZ\_2e2tau\_mll4\_2pt5			& {\sc Powheg+Pythia8} & 0.175  $\times$ 1 $\times$ 0.583 \\
126940	& ZZ\_4mu\_mll4\_2pt5				& {\sc Powheg+Pythia8} & 0.0769 $\times$ 1 $\times$ 0.912 \\
126941	& ZZ\_2mu2tau\_mll4\_2pt5			& {\sc Powheg+Pythia8} & 0.175  $\times$ 1 $\times$ 0.587 \\
126942	& ZZ\_4tau\_mll4\_2pt5				& {\sc Powheg+Pythia8} & 0.0769 $\times$ 1 $\times$ 0.106 \\
126949	& ZZllnunu\_ee\_mll4				& {\sc Powheg+Pythia8} & 0.0543 $\times$ 3 $\times$ 1 \\
126950	& ZZllnunu\_mm\_mll4				& {\sc Powheg+Pythia8} & 0.0543 $\times$ 3 $\times$ 1 \\
126951	& ZZllnunu\_tt\_mll4				& {\sc Powheg+Pythia8} & 0.0543 $\times$ 3 $\times$ 1 \\
129477	& WZ\_Wm11Z11\_mll0p250d0\_2LeptonFilter5	& {\sc Powheg+Pythia8} & 1.41  $\times$ 1.12 $\times$ 0.295 \\
129478	& WZ\_Wm11Z13\_mll0p4614d0\_2LeptonFilter5	& {\sc Powheg+Pythia8} & 0.938 $\times$ 1.12 $\times$ 0.352 \\
129479	& WZ\_Wm11Z15\_mll3p804d0\_2LeptonFilter5	& {\sc Powheg+Pythia8} & 0.175 $\times$ 1.12 $\times$ 0.167 \\
129480	& WZ\_Wm13Z11\_mll0p250d0\_2LeptonFilter5	& {\sc Powheg+Pythia8} & 1.40  $\times$ 1.12 $\times$ 0.294 \\
129481	& WZ\_Wm13Z13\_mll0p4614d0\_2LeptonFilter5	& {\sc Powheg+Pythia8} & 0.954 $\times$ 1.12 $\times$ 0.351 \\
129482	& WZ\_Wm13Z15\_mll3p804d0\_2LeptonFilter5	& {\sc Powheg+Pythia8} & 0.175 $\times$ 1.12 $\times$ 0.169 \\
129483	& WZ\_Wm15Z11\_mll0p250d0\_2LeptonFilter5	& {\sc Powheg+Pythia8} & 1.40  $\times$ 1.12 $\times$ 0.143 \\
129484	& WZ\_Wm15Z13\_mll0p4614d0\_2LeptonFilter5	& {\sc Powheg+Pythia8} & 0.938 $\times$ 1.12 $\times$ 0.183 \\
129485	& WZ\_Wm15Z15\_mll3p804d0\_2LeptonFilter5	& {\sc Powheg+Pythia8} & 0.172 $\times$ 1.12 $\times$ 0.0585 \\
129486	& WZ\_W11Z11\_mll0p250d0\_2LeptonFilter5	& {\sc Powheg+Pythia8} & 0.980 $\times$ 1.14 $\times$ 0.297 \\
129487	& WZ\_W11Z13\_mll0p4614d0\_2LeptonFilter5	& {\sc Powheg+Pythia8} & 0.639 $\times$ 1.14 $\times$ 0.353 \\
129488	& WZ\_W11Z15\_mll3p804d0\_2LeptonFilter5	& {\sc Powheg+Pythia8} & 0.113 $\times$ 1.14 $\times$ 0.160 \\
129489	& WZ\_W13Z11\_mll0p250d0\_2LeptonFilter5	& {\sc Powheg+Pythia8} & 0.936 $\times$ 1.14 $\times$ 0.298 \\
129490	& WZ\_W13Z13\_mll0p4614d0\_2LeptonFilter5	& {\sc Powheg+Pythia8} & 0.649 $\times$ 1.14 $\times$ 0.354 \\
129491	& WZ\_W13Z15\_mll3p804d0\_2LeptonFilter5	& {\sc Powheg+Pythia8} & 0.113 $\times$ 1.14 $\times$ 0.160 \\
129492	& WZ\_W15Z11\_mll0p250d0\_2LeptonFilter5	& {\sc Powheg+Pythia8} & 0.936 $\times$ 1.14 $\times$ 0.148 \\
129493	& WZ\_W15Z13\_mll0p4614d0\_2LeptonFilter5	& {\sc Powheg+Pythia8} & 0.639 $\times$ 1.14 $\times$ 0.187 \\
129494	& WZ\_W15Z15\_mll3p804d0\_2LeptonFilter5	& {\sc Powheg+Pythia8} & 0.111 $\times$ 1.14 $\times$ 0.0567 \\
\hline
\end{tabular}
%}
\caption{Summary table of the Monte Carlo samples used in the analysis. $\sigma$ is a LO cross-section times $k$-factor times event filter efficiency of $W/Z$ decay.}
\label{tab:mcsample_diboson2}
\end{center}
\end{table}
%%%%%%%%%%%%%%%%%%%%%%%%%%%%%%%%%%%%%%%%%%%%%%%%%%%%%%%%%%%%%%%%%%%%%%%%
% Z+JET (no filter)
%%%%%%%%%%%%%%%%%%%%%%%%%%%%%%%%%%%%%%%%%%%%%%%%%%%%%%%%%%%%%%%%%%%%%%%%
\begin{table}[htb]
\begin{center}
%\resizebox{\textwidth}{!}{
%\begin{tabular}{|c|c|c|c|}
\begin{tabular}{|p{1.7cm}|p{7.cm}|p{1.7cm}|p{3.8cm}|}
\hline
Sample ID 	& Name 					& Generator 		& $\sigma \times 10^3$[pb]\\
\hline\hline
167749 & ZeeMassiveCBPt0\_BFilter		& {\sc Sherpa} & 1.11 $\times$ 1.12 $\times$ 0.0280 \\
167750 & ZeeMassiveCBPt0\_CFilterBVeto		& {\sc Sherpa} & 1.11 $\times$ 1.12 $\times$ 0.283 \\
167751 & ZeeMassiveCBPt0\_CVetoBVeto		& {\sc Sherpa} & 1.11 $\times$ 1.12 $\times$ 0.686 \\
167752 & ZmumuMassiveCBPt0\_BFilter		& {\sc Sherpa} & 1.11 $\times$ 1.12 $\times$ 0.0280 \\
167753 & ZmumuMassiveCBPt0\_CFilterBVeto	& {\sc Sherpa} & 1.11 $\times$ 1.12 $\times$ 0.283 \\
167754 & ZmumuMassiveCBPt0\_CVetoBVeto		& {\sc Sherpa} & 1.11 $\times$ 1.12 $\times$ 0.690 \\
167755 & ZtautauMassiveCBPt0\_BFilter		& {\sc Sherpa} & 1.11 $\times$ 1.12 $\times$ 0.0278 \\
167756 & ZtautauMassiveCBPt0\_CFilterBVeto	& {\sc Sherpa} & 1.11 $\times$ 1.12 $\times$ 0.284 \\
167757 & ZtautauMassiveCBPt0\_CVetoBVeto	& {\sc Sherpa} & 1.11 $\times$ 1.12 $\times$ 0.689 \\
\hline
\end{tabular}
%}
\caption{Summary table of the Monte Carlo samples used in the analysis. $\sigma$ is a cross-section times
$k$-factor times event filter efficiency, which
is defined by the event fraction containing heavy-flavor ($b,c$) quarks and hadrons. 
These samples are used by combining with \pt\ sliced samples defined in Table\,\ref{tab:mcsample_zjet}, where events with $\pt(Z) > 40 \GeV$ are omitted
to avoid duplication.}
%%The definition can be found in {\tt Generators/GeneratorFilters/trunk/src/HeavyFlavorHadronFilter.cxx}.}
\label{tab:mcsample_zjet0}
\end{center}
\end{table}

%%%%%%%%%%%%%%%%%%%%%%%%%%%%%%%%%%%%%%%%%%%%%%%%%%%%%%%%%%%%%%%%%%%%%%%%
% Z+JET (pt slice)
%%%%%%%%%%%%%%%%%%%%%%%%%%%%%%%%%%%%%%%%%%%%%%%%%%%%%%%%%%%%%%%%%%%%%%%%
\begin{table}[htb]
\begin{center}
%\resizebox{\textwidth}{!}{
%\begin{tabular}{|c|c|c|c|}
\begin{tabular}{|p{1.7cm}|p{7.3cm}|p{1.7cm}|p{3.5cm}|}
\hline
Sample ID 	& Name 					& Generator 		& $\sigma$[pb]\\
\hline\hline
180543 & ZeeMassiveCBPt40\_70\_BFilter		& {\sc Sherpa} & 70.5 $\times$ 1.12 $\times$ 0.0706 \\
180544 & ZeeMassiveCBPt40\_70\_CFilterBVeto 	& {\sc Sherpa} & 70.5 $\times$ 1.12 $\times$ 0.342	 \\ 
180545 & ZeeMassiveCBPt40\_70\_CVetoBVeto 	& {\sc Sherpa} & 70.4 $\times$ 1.12 $\times$ 0.588	 \\ 
180546 & ZmumuMassiveCBPt40\_70\_BFilter 	& {\sc Sherpa} & 70.5 $\times$ 1.12 $\times$ 0.0707 \\ 
180547 & ZmumuMassiveCBPt40\_70\_CFilterBVeto 	& {\sc Sherpa} & 70.5 $\times$ 1.12 $\times$ 0.341	 \\ 
180548 & ZmumuMassiveCBPt40\_70\_CVetoBVeto 	& {\sc Sherpa} & 70.5 $\times$ 1.12 $\times$ 0.588	 \\ 
180549 & ZtautauMassiveCBPt40\_70\_BFilter 	& {\sc Sherpa} & 70.4 $\times$ 1.12 $\times$ 0.0709 \\ 
180550 & ZtautauMassiveCBPt40\_70\_CFilterBVeto & {\sc Sherpa} & 70.5 $\times$ 1.12 $\times$ 0.342	 \\ 
180551 & ZtautauMassiveCBPt40\_70\_CVetoBVeto 	& {\sc Sherpa} & 70.5 $\times$ 1.12 $\times$ 0.588	 \\ 
167797 & ZeeMassiveCBPt70\_140\_BFilter		& {\sc Sherpa} & 29.5 $\times$ 1.12 $\times$ 0.0825 \\
167798 & ZeeMassiveCBPt70\_140\_CFilterBVeto	& {\sc Sherpa} & 29.5 $\times$ 1.12 $\times$ 0.355 \\
167799 & ZeeMassiveCBPt70\_140\_CVetoBVeto	& {\sc Sherpa} & 29.5 $\times$ 1.12 $\times$ 0.563 \\
167800 & ZmumuMassiveCBPt70\_140\_BFilter	& {\sc Sherpa} & 29.5 $\times$ 1.12 $\times$ 0.0826 \\
167801 & ZmumuMassiveCBPt70\_140\_CFilterBVeto	& {\sc Sherpa} & 29.4 $\times$ 1.12 $\times$ 0.355 \\
167802 & ZmumuMassiveCBPt70\_140\_CVetoBVeto	& {\sc Sherpa} & 29.5 $\times$ 1.12 $\times$ 0.562 \\
167803 & ZtautauMassiveCBPt70\_140\_BFilter	& {\sc Sherpa} & 29.5 $\times$ 1.12 $\times$ 0.0826 \\
167804 & ZtautauMassiveCBPt70\_140\_CFilterBVeto& {\sc Sherpa} & 29.5 $\times$ 1.12 $\times$ 0.355 \\
167805 & ZtautauMassiveCBPt70\_140\_CVetoBVeto	& {\sc Sherpa} & 29.5 $\times$ 1.12 $\times$ 0.562 \\
167809 & ZeeMassiveCBPt140\_280\_BFilter	& {\sc Sherpa} & 3.99 $\times$ 1.12 $\times$ 0.0952 \\
167810 & ZeeMassiveCBPt140\_280\_CFilterBVeto	& {\sc Sherpa} & 3.98 $\times$ 1.12 $\times$ 0.369 \\
167811 & ZeeMassiveCBPt140\_280\_CVetoBVeto	& {\sc Sherpa} & 3.99 $\times$ 1.12 $\times$ 0.534 \\
167812 & ZmumuMassiveCBPt140\_280\_BFilter	& {\sc Sherpa} & 3.98 $\times$ 1.12 $\times$ 0.0954 \\
167813 & ZmumuMassiveCBPt140\_280\_CFilterBVeto	& {\sc Sherpa} & 3.99 $\times$ 1.12 $\times$ 0.370 \\
167814 & ZmumuMassiveCBPt140\_280\_CVetoBVeto	& {\sc Sherpa} & 3.98 $\times$ 1.12 $\times$ 0.534 \\
167815 & ZtautauMassiveCBPt140\_280\_BFilter	& {\sc Sherpa} & 3.99 $\times$ 1.12 $\times$ 0.0958 \\
167816 & ZtautauMassiveCBPt140\_280\_CFilterBVeto& {\sc Sherpa} & 3.99 $\times$ 1.12 $\times$ 0.370 \\
167817 & ZtautauMassiveCBPt140\_280\_CVetoBVeto	& {\sc Sherpa} & 3.99 $\times$ 1.12 $\times$ 0.533 \\
167821 & ZeeMassiveCBPt280\_500\_BFilter	& {\sc Sherpa} & 0.242 $\times$ 1.12 $\times$ 0.109 \\
167822 & ZeeMassiveCBPt280\_500\_CFilterBVeto	& {\sc Sherpa} & 0.241 $\times$ 1.12 $\times$ 0.387 \\
167823 & ZeeMassiveCBPt280\_500\_CVetoBVeto	& {\sc Sherpa} & 0.242 $\times$ 1.12 $\times$ 0.506 \\
167824 & ZmumuMassiveCBPt280\_500\_BFilter	& {\sc Sherpa} & 0.242 $\times$ 1.12 $\times$ 0.108 \\
167825 & ZmumuMassiveCBPt280\_500\_CFilterBVeto	& {\sc Sherpa} & 0.242 $\times$ 1.12 $\times$ 0.386 \\
167826 & ZmumuMassiveCBPt280\_500\_CVetoBVeto	& {\sc Sherpa} & 0.243 $\times$ 1.12 $\times$ 0.505 \\
167827 & ZtautauMassiveCBPt280\_500\_BFilter	& {\sc Sherpa} & 0.241 $\times$ 1.12 $\times$ 0.107 \\
167828 & ZtautauMassiveCBPt280\_500\_CFilterBVeto& {\sc Sherpa} & 0.241 $\times$ 1.12 $\times$ 0.385 \\
167829 & ZtautauMassiveCBPt280\_500\_CVetoBVeto	& {\sc Sherpa} & 0.241 $\times$ 1.12 $\times$ 0.507 \\
167833 & ZeeMassiveCBPt500\_BFilter		& {\sc Sherpa} & 0.0132 $\times$ 1.12 $\times$ 0.116 \\
167834 & ZeeMassiveCBPt500\_CFilterBVeto	& {\sc Sherpa} & 0.0135 $\times$ 1.12 $\times$ 0.398 \\
167835 & ZeeMassiveCBPt500\_CVetoBVeto		& {\sc Sherpa} & 0.0133 $\times$ 1.12 $\times$ 0.485 \\
167836 & ZmumuMassiveCBPt500\_BFilter		& {\sc Sherpa} & 0.0132 $\times$ 1.12 $\times$ 0.114 \\
167837 & ZmumuMassiveCBPt500\_CFilterBVeto	& {\sc Sherpa} & 0.0135 $\times$ 1.12 $\times$ 0.399 \\
167838 & ZmumuMassiveCBPt500\_CVetoBVeto	& {\sc Sherpa} & 0.0133 $\times$ 1.12 $\times$ 0.487 \\
167839 & ZtautauMassiveCBPt500\_BFilter		& {\sc Sherpa} & 0.0132 $\times$ 1.12 $\times$ 0.115 \\
167840 & ZtautauMassiveCBPt500\_CFilterBVeto	& {\sc Sherpa} & 0.0133 $\times$ 1.12 $\times$ 0.393 \\
167841 & ZtautauMassiveCBPt500\_CVetoBVeto	& {\sc Sherpa} & 0.0133 $\times$ 1.12 $\times$ 0.486 \\
\hline
\end{tabular}
%}
\caption{Summary table of the Monte Carlo samples used in the analysis. $\sigma$ is a cross-section times
$k$-factor times event filter efficiency, which
is defined by the event fraction containing heavy-flavor ($b,c$) quarks and hadrons.}
\label{tab:mcsample_zjet}
\end{center}
\end{table}

%%%%%%%%%%%%%%%%%%%%%%%%%%%%%%%%%%%%%%%%%%%%%%%%%%%%%%%%%%%%%%%%%%%%%%%%
% Z+jets HT filtered
%%%%%%%%%%%%%%%%%%%%%%%%%%%%%%%%%%%%%%%%%%%%%%%%%%%%%%%%%%%%%%%%%%%%%%%%
\begin{table}[htb]
\begin{center}
%\resizebox{\textwidth}{!}{
%\begin{tabular}{|c|c|c|c|}
\begin{tabular}{|p{1.7cm}|p{7.cm}|p{1.7cm}|p{3.8cm}|}
\hline
Sample ID 	& Name 					& Generator 		& $\sigma$[pb]\\
\hline\hline
204659 & ZeeMassiveCBPt0\_40\_HT500		    & {\sc Sherpa} & 1006 $\times$ 1.12 $\times$ 3.75$\cdot 10^{-4}$ \\
204660 & ZeeMassiveCBPt40\_70\_HT500		& {\sc Sherpa} & 70.5 $\times$ 1.12 $\times$ 4.36$\cdot 10^{-3}$\\
204661 & ZeeMassiveCBPt70\_140\_HT500		& {\sc Sherpa} & 29.5 $\times$ 1.12 $\times$ 3.30$\cdot 10^{-2}$ \\
204662 & ZeeMassiveCBPt140\_280\_HT500	    & {\sc Sherpa} & 4.01 $\times$ 1.12 $\times$ 0.632 \\
204663 & ZmumuMassiveCBPt0\_40\_HT500	 	& {\sc Sherpa} & 1110 $\times$ 1.12 $\times$ 1.20$\cdot 10^{-3}$ \\
204664 & ZmumuMassiveCBPt40\_70\_HT500		& {\sc Sherpa} & 70.5 $\times$ 1.12 $\times$ 1.85$\cdot 10^{-3}$ \\
204665 & ZmumuMassiveCBPt70\_140\_HT500		& {\sc Sherpa} & 29.5 $\times$ 1.12 $\times$ 1.02$\cdot 10^{-2}$ \\
204666 & ZmumuMassiveCBPt140\_280\_HT500	& {\sc Sherpa} & 3.99 $\times$ 1.12 $\times$ 0.137 \\
\hline
\end{tabular}
%}
\caption{Summary table of the Monte Carlo samples used in the analysis. $\sigma$ is a cross-section times
$k$-factor times event filter efficiency, which
is defined by applying a cut on truth $\HT > 500~\GeV$ to gain statistics in the high-\HT , high-\met\ regions.
These samples are used by combining them with \pt\ sliced samples defined in Tables~\ref{tab:mcsample_zjet0} and~\ref{tab:mcsample_zjet}, for which only events with truth $\HT < 500~\GeV$ are selected in the $ee$ and $\mu\mu$ channel.}
%%The definition can be found in {\tt Generators/GeneratorFilters/trunk/src/HeavyFlavorHadronFilter.cxx}.}
\label{tab:mcsample_HTFilter}
\end{center}
\end{table}







%%%%%%%%%%%%%%%%%%%%%%%%%%%%%%%%%%%%%%%%%%%%%%%%%%%%%%%%%%%%%%%%%%%%%%%%
% DY
%%%%%%%%%%%%%%%%%%%%%%%%%%%%%%%%%%%%%%%%%%%%%%%%%%%%%%%%%%%%%%%%%%%%%%%%
\begin{table}[htb]
\begin{center}
%\resizebox{\textwidth}{!}{
%\begin{tabular}{|c|c|c|c|}
\begin{tabular}{|p{1.7cm}|p{5cm}|p{4.5cm}|p{3cm}|}
\hline
Sample ID 	& Name 					& Generator 		& $\sigma$ [pb]\\
\hline\hline
173041  & DYeeM08to15                 & {\sc Sherpa} & 92.2 \\
173042  & DYeeM15to40                 & {\sc Sherpa} & 279 \\
173043  & DYmumuM08to15               & {\sc Sherpa} & 92.1 \\
173044  & DYmumuM15to40               & {\sc Sherpa} & 279 \\
173045  & DYtautauM08to15             & {\sc Sherpa} & 92.1 \\
173046  & DYtautauM15to40             & {\sc Sherpa} & 279 \\
\hline
\end{tabular}
%}
\caption{Summary table of the Monte Carlo samples used in the analysis. $\sigma$ is a cross-section times $k$-factor.}
\label{tab:mcsample_dy}
\end{center}
\end{table}


%\begin{itemize}
%\item $t\bar{t}$:
%For \ttbar\ simulation Powheg~\cite{powheg1} connected to Pythia~\cite{pythia} for the parton shower simulation is used.

%\item \dyjets and \wjets: 
%The ALPGEN~\cite{alpgen} generator connected to Herwig/Jimmy~\cite{herwig} is used for \dyjets and \wjets simulation.

%\item Diboson:
%Diboson samples are generated using MC@NLO for WW and Herwig~\cite{herwig} for WZ, ZZ.

%\item Single $t$: 
%Single top samples are generated using Acer~(t-channel) and MC@NLO~(s-channel).

%\item $t\bar{t} + V$: MadGraphPythia
%\ttbar+V is generated with MadGraph~\cite{Alwall:2007st} for the hard scattering,
%and Pythia~\cite{pythia} for the parton shower simulation.

%\item QCD: The QCD background is determined from data using a matrix method.
%\end{itemize}

%All Monte Carlo samples used in this analyses are from the official MC12a
%Monte Carlo production. These samples use the production tag \texttt{p1328} (CHECK). 
%The list of simulated background samples used in this analysis, including
% the cross-sections, filter efficiencies, the generators, and the $k$-factors, 
% is listed in in COMBINATION INT NOTE. 
% %Tables~\ref{tab:mcsamples} and ~\ref{tab:mcsamples2}.
% Note that all cross sections, NLO or NNLO k-factors and
%  filter efficiencies are obtained from the recommended list provided by
%  the SUSY group. 


%  https://svnweb.cern.ch/trac/atlasoff/browser/PhysicsAnalysis/SUSYPhys/SUSYTools/trunk/
%  data/susy\_crosssections\_8TeV.txt

%%%%%%%% $t\bar{t}$: PowhegPythia


%%%%%%%% \dyjets: AlpgenJimmy

%%%%%%%% \item \wjets: AlpgenJimmy

%%%%%%%% \item Diboson: MCAtNLOJimmy and Herwig

%%%%%%%% \item Single $t$: AcerMCPythia and MCAtNLOJimmy

%%%%%%%% \item $t\bar{t} + V$: MadGraphPythia

%%%%%%%% \item QCD: Matrix method	



%%%%%%%%%%%%%%%%%%%%%%%%%%%%%%%%%%%%%%%%%%%%%%%%%%%%%%%%%%%%%%%%%%%%%%%%%%%%%

\subsection{General gauge mediation signal samples}
\label{sec:GGM-samples}

For the GGM models the SUSY mass spectra, 
the gluino branching ratios, and the gluino 
decay width are calculated using {\sc SUSPECT~2.41}~\cite{Djouadi:2002ze} 
and {\sc SDECAY~1.3}~\cite{Muhlleitner:2003vg}.
Table~\ref{tab:mc:par} shows the values of the parameters
used for the signal MC production. 
All other sparticle masses are fixed at $\sim$1.5\TeV.

\begin{table}[hbp]
\centering
\begin{tabular}{ccccccc}
\hline
Parameters
& $M_{1}$
& $M_{2}$
& $\tan\beta$
& $c\tau_{\text{NLSP}}$
& $\mu$
& $\mass{\gluino}$ \\
\hline
\hline
Values     & $\unit[1]{TeV}$ & $\unit[1]{TeV}$ & 1.5 & $<\unit[0.1]{mm}$ & vary & vary \\
\hline
Values     & $\unit[1]{TeV}$ & $\unit[1]{TeV}$ & 30  & $<\unit[0.1]{mm}$ & vary & vary \\
\hline
\end{tabular}
\caption{Parameters of the GGM models used for the signal production. The constraint $c\tau_{\text{NLSP}}<\unit[0.1]{mm}$ is fulfilled for all input values 
     of $\mu > 120$~\gev.}
\label{tab:mc:par}
\end{table}

The MC signal samples are generated using 
{\sc PYTHIA~6.423}~\cite{Sjostrand:2006za} with
{\sc MRST2007} LO$^{*}$~\cite{Sherstnev:2007nd} 
parton distribution functions (PDF). All samples
are listed in Appendix~\ref{sec:GGM_points}.  Signal cross 
sections are calculated to next-to-leading order (NLO) 
in the strong coupling constant, adding the resummation 
of soft gluon emission at next-to-leading-logarithmic 
accuracy (NLO+NLL)~\cite{Beenakker:1996ch,Kulesza:2008jb,Kulesza:2009kq,Beenakker:2009ha,Beenakker:2011fu}.
The nominal cross-section and the uncertainty are taken from an envelope
of cross-section predictions using different PDF sets and factorisation 
and renormalisation scales, as described in Ref.~\cite{Kramer:2012bx}.
The GGM signal datasets are listed in Tables~\ref{tab:mcsample_tanbeta15} and~\ref{tab:mcsample_tanbeta30}.

\begin{figure}[hbp]
\centering
\includegraphics[width=3in]{Datasets/fig/ggm/SP.eps}
\caption{The fraction of the \gluino\gluino~ processes as function of
$\mu$ for GGM grid models characterised by the following parameters:
$M_{1} = 1\TeV,\ M_{2} = 1\TeV,\ \mass{\gluino} = 900\GeV$,
$\tan\beta$ = 1.5.}
\label{fig:sp}
\end{figure}

The mechanisms for sparticle production in 
the signal points can be grouped into two 
categories: strong production (\gluino\gluino), 
and electroweak production ($\tilde{\chi}^{0}_{1}\tilde{\chi}^{0}_{2}$,
$\tilde{\chi}^{\pm}_{1}\tilde{\chi}^{\mp}_{1}$, $\tilde{\chi}^{0}_{1}\tilde{\chi}^{\pm}_{1}$,
$\tilde{\chi}^{0}_{2}\tilde{\chi}^{\pm}_{1}$). The contribution from 
strong production as function of $\mu$ is presented in Figure~\ref{fig:sp} for
GGM grid models characterised by the following parameters:
$M_{1} = 1\TeV,\ M_{2} = 1\TeV,\ \mass{\gluino} = 900\GeV$, $\tan\beta$ = 1.5.
All processes are considered for NLO and NNL calculations.
 
To select events with at least one $Z$ boson that decays to a 
pair of electrons, muons, or taus, the \texttt{ZtoLeptonFilter}
is used at generator level. This filter has an efficiency 
of $18.8 \pm 0.5 \%$ for models with $\tan\beta = 1.5$.
The filter efficiency for the models with $\tan\beta = 30$ 
is dependent on $\mu$, because $BR(\neutralino\to h\gravitino)$ increases with
$\mu$ for these models, reaching a value of 0.39 for $\mu = 790\GeV$, as shown in Figure~\ref{fig:br} (right). 
The $BR(\neutralino\to h\gravitino)$ 
and $BR(\neutralino\to Z\gravitino)$ for $\tan\beta = 1.5$ 
is also presented at Figure~\ref{fig:br} (left).
Tables~\ref{tab:cs1},~\ref{tab:cs1_1},~\ref{tab:cs2}, and ~\ref{tab:cs3}
present $\mu$, masses of $\tilde{\chi}^{0}_{1}$, $\tilde{\chi}^{0}_{2}$,
$\tilde{\chi}^{\pm}_{1}$, the NLO+NLL cross-section, the filter efficiency,
and the effective NLO+NLL cross-section for each generated GGM point.
Table~\ref{tab:mc:cs4} presents the NLO+NLL cross-section for
$\gluino\gluino$ production as a function of $\mass{\gluino}$.
$\gluino\gluino$ production is expected to have no dependence on $\mu$.
\begin{figure}[hbp]
\centering
\thesubfigure{\includegraphics[width=3in]{Datasets/fig/ggm/BR15.eps}}
\thesubfigure{\includegraphics[width=3in]{Datasets/fig/ggm/BR.eps}}
\caption{The branching fraction for $\neutralino\to h\gravitino$ and
$\neutralino\to Z\gravitino$ processes as function of 
$\mu$ for GGM grid models characterised by the following parameters:
$M_{1} = 1\TeV,\ M_{2} = 1\TeV,\ \mass{\gluino} = 800\GeV$,
$\tan\beta = 1.5$ (left), $\tan\beta = 30$ (right).}
\label{fig:br}
\end{figure}


\begin{table}[hbp] 
\centering
\begin{tabular}{cc} 
\hline  
$\mass{\gluino} \ [GeV]$ & 
(NLO+NLL) $\sigma(\gluino \gluino) \ [pb]$ \\
\hline \hline
400 & 18.2 $\pm$ 2.8 \\
500 & 4.2 $\pm$ 0.7 \\
600 & 1.2 $\pm$ 0.2 \\
700 & 0.37 $\pm$ 0.08 \\
800 & 0.13$\pm$0.03 \\
900 & 0.046$\pm$0.014 \\
1000 & 0.017$\pm$0.006 \\
\hline
\end{tabular}
\caption{The NLO+NLL cross-sections for \gluino\gluino \ production only.}
\label{tab:mc:cs4}
\end{table}

%In total, XXX GGM signal samples were generated. 
As an example of GGM signal samples generated, 
the full SUSY particle mass spectrum for the model with $\tan\beta=1.5$, 
$\mass{\gluino} = 700\GeV$ and $\mu = 200\GeV$
is listed in Table~\ref{tab:mc:sp}.

\begin{table}[hbp]
\centering
\begin{tabular}{lcc}
\hline
Names & Mass Eigenstates & m $[\GeV]$ \\
\hline \hline
\multirow{12}{*}{squarks}
& $\tilde{u}_{L}$ & \multirow{12}{*}{$\approx$ 1500}  \\
& $\tilde{u}_{R}$ &  \\
& $\tilde{d}_{L}$ &  \\
& $\tilde{d}_{R}$ & \\
& $\tilde{s}_{L}$ & \\
& $\tilde{s}_{R}$ & \\
& $\tilde{c}_{L}$ & \\
& $\tilde{c}_{R}$ &  \\
& $\tilde{t}_{1}$ &   \\
& $\tilde{t}_{2}$ &   \\
& $\tilde{b}_{1}$ &   \\
& $\tilde{b}_{2}$ &   \\
\hline
\multirow{6}{*}{sleptons}
& $\tilde{e}_{L}$       &  \multirow{6}{*}{$\approx$ 1500}\\
& $\tilde{e}_{R}$       &  \\
& $\tilde{\mu}_{L}$     &  \\
& $\tilde{\mu}_{R}$     &  \\
& $\tilde{\tau}_{1}$    &  \\
& $\tilde{\tau}_{2}$    &  \\
\hline
\hline
\multirow{4}{*}{neutralinos}   
& $\tilde{\chi}^{0}_{1}$     &  190.1   \\
& $\tilde{\chi}^{0}_{2}$     &  200.3  \\
& $\tilde{\chi}^{0}_{3}$     &  1000.0  \\
& $\tilde{\chi}^{0}_{4}$     &  1010.2  \\
\hline
\multirow{2}{*}{charginos}
& $\tilde{\chi}^{\pm}_{1}$      & 192.6   \\
& $\tilde{\chi}^{\pm}_{2}$      & 1007.8  \\
\hline
\multirow{4}{*}{Higgs bosons}
& $\tilde{h}^{0}$            & 126.0  \\
& $\tilde{H}^{0}$            & 2004.0   \\
& $\tilde{A}^{0}$            & 2000.0   \\
& $\tilde{H}^{\pm}$       & 2002.1   \\
\hline
\end{tabular}
\caption{SUSY particle mass spectrum for the higgsino-like NLSP GGM model
which is characterised by the following parameters:
$M_{1} = 1\TeV,\ M_{2} = 1\TeV,\  \tan\beta = 1.5,\ 
c\tau_\text{NLSP} < 0.1~\text{mm},\
\mu= 200\GeV,\ \mass{\gluino} = 700\GeV$.}
\label{tab:mc:sp}
\end{table}

%%%%%%%%%%% tan beta 1.5 %%%%%%%%%%%%
\begin{table}[htb]
\begin{center}
\scriptsize
%\resizebox{\textwidth}{!}{
%\begin{tabular}{|c|c|c|c|}
\begin{tabular}{|p{1.7cm}|p{7cm}|p{2.5cm}|}
\hline
Sample ID 	& Name 					& Generator \\
\hline\hline
 142585  &  GGMHiggsinoNLSP\_800\_120\_Zll  &  {\sc Pythia} \\ 
 142586  &  GGMHiggsinoNLSP\_800\_150\_Zll  &  {\sc Pythia} \\ 
 142587  &  GGMHiggsinoNLSP\_800\_200\_Zll  &  {\sc Pythia} \\ 
 142588  &  GGMHiggsinoNLSP\_800\_300\_Zll  &  {\sc Pythia} \\ 
 142589  &  GGMHiggsinoNLSP\_800\_400\_Zll  &  {\sc Pythia} \\ 
 142590  &  GGMHiggsinoNLSP\_800\_500\_Zll  &  {\sc Pythia} \\ 
 142591  &  GGMHiggsinoNLSP\_800\_600\_Zll  &  {\sc Pythia} \\ 
 142592  &  GGMHiggsinoNLSP\_800\_700\_Zll  &  {\sc Pythia} \\ 
 142593  &  GGMHiggsinoNLSP\_800\_790\_Zll  &  {\sc Pythia} \\ 
 142594  &  GGMHiggsinoNLSP\_900\_120\_Zll  &  {\sc Pythia} \\ 
 142595  &  GGMHiggsinoNLSP\_900\_150\_Zll  &  {\sc Pythia} \\ 
 142596  &  GGMHiggsinoNLSP\_900\_200\_Zll  &  {\sc Pythia} \\ 
 142597  &  GGMHiggsinoNLSP\_900\_300\_Zll  &  {\sc Pythia} \\ 
 142598  &  GGMHiggsinoNLSP\_900\_400\_Zll  &  {\sc Pythia} \\ 
 142599  &  GGMHiggsinoNLSP\_900\_500\_Zll  &  {\sc Pythia} \\ 
 142599  &  GGMHiggsinoNLSP\_900\_500\_Zll  &  {\sc Pythia} \\
 142600  &  GGMHiggsinoNLSP\_900\_600\_Zll  &  {\sc Pythia} \\ 
 142601  &  GGMHiggsinoNLSP\_900\_700\_Zll  &  {\sc Pythia} \\ 
 142602  &  GGMHiggsinoNLSP\_900\_800\_Zll  &  {\sc Pythia} \\ 
 142603  &  GGMHiggsinoNLSP\_900\_890\_Zll  &  {\sc Pythia} \\ 
 164424  &  GGMHiggsinoNLSP\_1000\_120\_Zll  &  {\sc Pythia} \\ 
 164425  &  GGMHiggsinoNLSP\_1000\_150\_Zll  &  {\sc Pythia} \\ 
 164426  &  GGMHiggsinoNLSP\_1000\_200\_Zll  &  {\sc Pythia} \\ 
 164427  &  GGMHiggsinoNLSP\_1000\_300\_Zll  &  {\sc Pythia} \\ 
 164428  &  GGMHiggsinoNLSP\_1000\_400\_Zll  &  {\sc Pythia} \\ 
 164429  &  GGMHiggsinoNLSP\_1000\_500\_Zll  &  {\sc Pythia} \\ 
 164430  &  GGMHiggsinoNLSP\_1000\_600\_Zll  &  {\sc Pythia} \\ 
 164431  &  GGMHiggsinoNLSP\_1000\_700\_Zll  &  {\sc Pythia} \\ 
 164432  &  GGMHiggsinoNLSP\_1000\_800\_Zll  &  {\sc Pythia} \\ 
 164433  &  GGMHiggsinoNLSP\_1000\_900\_Zll  &  {\sc Pythia} \\ 
 164434  &  GGMHiggsinoNLSP\_1000\_990\_Zll  &  {\sc Pythia} \\ 
 164897  &  GGMHiggsinoNLSP\_600\_120\_Zll  &  {\sc Pythia} \\ 
 164898  &  GGMHiggsinoNLSP\_600\_150\_Zll  &  {\sc Pythia} \\ 
 164899  &  GGMHiggsinoNLSP\_600\_200\_Zll  &  {\sc Pythia} \\ 
 164900  &  GGMHiggsinoNLSP\_600\_300\_Zll  &  {\sc Pythia} \\ 
 164901  &  GGMHiggsinoNLSP\_600\_400\_Zll  &  {\sc Pythia} \\ 
 164902  &  GGMHiggsinoNLSP\_600\_500\_Zll  &  {\sc Pythia} \\ 
 164903  &  GGMHiggsinoNLSP\_600\_590\_Zll  &  {\sc Pythia} \\ 
 164904  &  GGMHiggsinoNLSP\_700\_120\_Zll  &  {\sc Pythia} \\ 
 164905  &  GGMHiggsinoNLSP\_700\_150\_Zll  &  {\sc Pythia} \\ 
 164906  &  GGMHiggsinoNLSP\_700\_200\_Zll  &  {\sc Pythia} \\ 
 164907  &  GGMHiggsinoNLSP\_700\_300\_Zll  &  {\sc Pythia} \\ 
 164908  &  GGMHiggsinoNLSP\_700\_400\_Zll  &  {\sc Pythia} \\ 
 164909  &  GGMHiggsinoNLSP\_700\_500\_Zll  &  {\sc Pythia} \\ 
 164910  &  GGMHiggsinoNLSP\_700\_600\_Zll  &  {\sc Pythia} \\ 
 164911  &  GGMHiggsinoNLSP\_700\_690\_Zll  &  {\sc Pythia} \\ 
 176216  &  GGMHiggsinoNLSP\_1100\_120\_Zll & {\sc Pythia} \\ 
 176217  &  GGMHiggsinoNLSP\_1100\_150\_Zll & {\sc Pythia} \\ 
 176218  &  GGMHiggsinoNLSP\_1100\_200\_Zll & {\sc Pythia} \\ 
 176219  &  GGMHiggsinoNLSP\_1100\_300\_Zll & {\sc Pythia} \\ 
 176220  &  GGMHiggsinoNLSP\_1100\_400\_Zll & {\sc Pythia} \\ 
 176221  &  GGMHiggsinoNLSP\_1100\_500\_Zll & {\sc Pythia} \\ 
 176222  &  GGMHiggsinoNLSP\_1100\_600\_Zll & {\sc Pythia} \\ 
 176223  &  GGMHiggsinoNLSP\_1100\_700\_Zll & {\sc Pythia} \\ 
 176224  &  GGMHiggsinoNLSP\_1100\_800\_Zll & {\sc Pythia} \\ 
 176225  &  GGMHiggsinoNLSP\_1100\_900\_Zll & {\sc Pythia} \\ 
 176226  &  GGMHiggsinoNLSP\_1100\_1000\_Zll & {\sc Pythia} \\ 
 176227  &  GGMHiggsinoNLSP\_1100\_1090\_Zll & {\sc Pythia} \\ 
 176228  &  GGMHiggsinoNLSP\_1200\_120\_Zll & {\sc Pythia} \\ 
 176229  &  GGMHiggsinoNLSP\_1200\_150\_Zll & {\sc Pythia} \\ 
 176230  &  GGMHiggsinoNLSP\_1200\_200\_Zll & {\sc Pythia} \\ 
 176231  &  GGMHiggsinoNLSP\_1200\_300\_Zll & {\sc Pythia} \\ 
 176232  &  GGMHiggsinoNLSP\_1200\_400\_Zll & {\sc Pythia} \\ 
 176233  &  GGMHiggsinoNLSP\_1200\_500\_Zll & {\sc Pythia} \\ 
 176234  &  GGMHiggsinoNLSP\_1200\_600\_Zll & {\sc Pythia} \\ 
 176235  &  GGMHiggsinoNLSP\_1200\_700\_Zll & {\sc Pythia} \\ 
 176236  &  GGMHiggsinoNLSP\_1200\_800\_Zll & {\sc Pythia} \\ 
 176237  &  GGMHiggsinoNLSP\_1200\_900\_Zll & {\sc Pythia} \\ 
 176238  &  GGMHiggsinoNLSP\_1200\_1000\_Zll & {\sc Pythia} \\ 
 176239  &  GGMHiggsinoNLSP\_1200\_1100\_Zll & {\sc Pythia} \\ 
 176240  &  GGMHiggsinoNLSP\_1200\_1190\_Zll & {\sc Pythia} \\ 
\hline
\end{tabular}
\caption{Summary table of the GGM Monte Carlo samples used in the analysis with $\tan{\beta}=1.5$.}
\label{tab:mcsample_tanbeta15}
\end{center}
\end{table}
%%%%%%%%%%% tan beta 3.0 %%%%%%%%%%%%
\begin{table}[htb]
\begin{center}
\tiny
%\resizebox{\textwidth}{!}{
%\begin{tabular}{|c|c|c|c|}
\begin{tabular}{|p{1.7cm}|p{7cm}|p{2.5cm}|}
%\begin{longtable}{|p{1.7cm}|p{7cm}|p{2.5cm}|}
\hline
Sample ID 	& Name 					& Generator\\
\hline\hline
 176241  &  GGMHiggsinoNLSP\_1100\_120\_30\_Zll & {\sc Pythia} \\ 
 176242  &  GGMHiggsinoNLSP\_1100\_150\_30\_Zll & {\sc Pythia} \\ 
 176243  &  GGMHiggsinoNLSP\_1100\_200\_30\_Zll & {\sc Pythia} \\ 
 176244  &  GGMHiggsinoNLSP\_1100\_300\_30\_Zll & {\sc Pythia} \\ 
 176245  &  GGMHiggsinoNLSP\_1100\_400\_30\_Zll & {\sc Pythia} \\ 
 176246  &  GGMHiggsinoNLSP\_1100\_500\_30\_Zll & {\sc Pythia} \\ 
 176247  &  GGMHiggsinoNLSP\_1100\_600\_30\_Zll & {\sc Pythia} \\ 
 176248  &  GGMHiggsinoNLSP\_1100\_700\_30\_Zll & {\sc Pythia} \\ 
 176249  &  GGMHiggsinoNLSP\_1100\_800\_30\_Zll & {\sc Pythia} \\ 
 176250  &  GGMHiggsinoNLSP\_1100\_900\_30\_Zll & {\sc Pythia} \\ 
 176251  &  GGMHiggsinoNLSP\_1100\_1000\_30\_Zll & {\sc Pythia} \\ 
 176252  &  GGMHiggsinoNLSP\_1100\_1090\_30\_Zll & {\sc Pythia} \\ 
 176253  &  GGMHiggsinoNLSP\_1200\_120\_30\_Zll & {\sc Pythia} \\ 
 176254  &  GGMHiggsinoNLSP\_1200\_150\_30\_Zll & {\sc Pythia} \\ 
 176255  &  GGMHiggsinoNLSP\_1200\_200\_30\_Zll & {\sc Pythia} \\ 
 176256  &  GGMHiggsinoNLSP\_1200\_300\_30\_Zll & {\sc Pythia} \\ 
 176257  &  GGMHiggsinoNLSP\_1200\_400\_30\_Zll & {\sc Pythia} \\ 
 176258  &  GGMHiggsinoNLSP\_1200\_500\_30\_Zll & {\sc Pythia} \\ 
 176259  &  GGMHiggsinoNLSP\_1200\_600\_30\_Zll & {\sc Pythia} \\ 
 176260  &  GGMHiggsinoNLSP\_1200\_700\_30\_Zll & {\sc Pythia} \\ 
 176261  &  GGMHiggsinoNLSP\_1200\_800\_30\_Zll & {\sc Pythia} \\ 
 176262  &  GGMHiggsinoNLSP\_1200\_900\_30\_Zll & {\sc Pythia} \\ 
 176263  &  GGMHiggsinoNLSP\_1200\_1000\_30\_Zll & {\sc Pythia} \\ 
 176264  &  GGMHiggsinoNLSP\_1200\_1100\_30\_Zll & {\sc Pythia} \\ 
 176265  &  GGMHiggsinoNLSP\_1200\_1190\_30\_Zll & {\sc Pythia} \\ 
 157572  &  GGMHiggsinoNLSP\_400\_120\_30\_Zll  &  {\sc Pythia}  \\  
 157573  &  GGMHiggsinoNLSP\_400\_150\_30\_Zll  &  {\sc Pythia}  \\  
 157574  &  GGMHiggsinoNLSP\_400\_200\_30\_Zll  &  {\sc Pythia}  \\  
 157575  &  GGMHiggsinoNLSP\_400\_300\_30\_Zll  &  {\sc Pythia}  \\  
 157576  &  GGMHiggsinoNLSP\_400\_390\_30\_Zll  &  {\sc Pythia}  \\  
 157577  &  GGMHiggsinoNLSP\_500\_120\_30\_Zll  &  {\sc Pythia}  \\  
 157578  &  GGMHiggsinoNLSP\_500\_150\_30\_Zll  &  {\sc Pythia}  \\  
 157579  &  GGMHiggsinoNLSP\_500\_200\_30\_Zll  &  {\sc Pythia}  \\  
 157580  &  GGMHiggsinoNLSP\_500\_300\_30\_Zll  &  {\sc Pythia}  \\  
 157581  &  GGMHiggsinoNLSP\_500\_400\_30\_Zll  &  {\sc Pythia}  \\  
 157582  &  GGMHiggsinoNLSP\_500\_490\_30\_Zll  &  {\sc Pythia}  \\  
 157583  &  GGMHiggsinoNLSP\_600\_120\_30\_Zll  &  {\sc Pythia}  \\  
 157584  &  GGMHiggsinoNLSP\_600\_150\_30\_Zll  &  {\sc Pythia}  \\  
 157585  &  GGMHiggsinoNLSP\_600\_200\_30\_Zll  &  {\sc Pythia}  \\  
 157586  &  GGMHiggsinoNLSP\_600\_300\_30\_Zll  &  {\sc Pythia}  \\  
 157587  &  GGMHiggsinoNLSP\_600\_400\_30\_Zll  &  {\sc Pythia}  \\  
 157588  &  GGMHiggsinoNLSP\_600\_500\_30\_Zll  &  {\sc Pythia}  \\  
 157589  &  GGMHiggsinoNLSP\_600\_590\_30\_Zll  &  {\sc Pythia}  \\  
 157590  &  GGMHiggsinoNLSP\_700\_120\_30\_Zll  &  {\sc Pythia}  \\  
 157591  &  GGMHiggsinoNLSP\_700\_150\_30\_Zll  &  {\sc Pythia}  \\  
 157592  &  GGMHiggsinoNLSP\_700\_200\_30\_Zll  &  {\sc Pythia}  \\  
 157593  &  GGMHiggsinoNLSP\_700\_300\_30\_Zll  &  {\sc Pythia}  \\  
 157594  &  GGMHiggsinoNLSP\_700\_400\_30\_Zll  &  {\sc Pythia}  \\  
 157595  &  GGMHiggsinoNLSP\_700\_500\_30\_Zll  &  {\sc Pythia}  \\  
 157596  &  GGMHiggsinoNLSP\_700\_600\_30\_Zll  &  {\sc Pythia}  \\  
 157597  &  GGMHiggsinoNLSP\_700\_690\_30\_Zll  &  {\sc Pythia}  \\  
 157598  &  GGMHiggsinoNLSP\_800\_120\_30\_Zll  &  {\sc Pythia}  \\  
 157599  &  GGMHiggsinoNLSP\_800\_150\_30\_Zll  &  {\sc Pythia}  \\  
 157600  &  GGMHiggsinoNLSP\_800\_200\_30\_Zll  &  {\sc Pythia}  \\  
 157601  &  GGMHiggsinoNLSP\_800\_300\_30\_Zll  &  {\sc Pythia}  \\  
 157602  &  GGMHiggsinoNLSP\_800\_400\_30\_Zll  &  {\sc Pythia}  \\  
 157603  &  GGMHiggsinoNLSP\_800\_500\_30\_Zll  &  {\sc Pythia}  \\  
 157604  &  GGMHiggsinoNLSP\_800\_600\_30\_Zll  &  {\sc Pythia}  \\  
 157605  &  GGMHiggsinoNLSP\_800\_700\_30\_Zll  &  {\sc Pythia}  \\  
 157606  &  GGMHiggsinoNLSP\_800\_790\_30\_Zll  &  {\sc Pythia}  \\  
 157607  &  GGMHiggsinoNLSP\_900\_120\_30\_Zll  &  {\sc Pythia}  \\  
 157608  &  GGMHiggsinoNLSP\_900\_150\_30\_Zll  &  {\sc Pythia}  \\  
 157609  &  GGMHiggsinoNLSP\_900\_200\_30\_Zll  &  {\sc Pythia}  \\  
 157610  &  GGMHiggsinoNLSP\_900\_300\_30\_Zll  &  {\sc Pythia}  \\  
 157611  &  GGMHiggsinoNLSP\_900\_400\_30\_Zll  &  {\sc Pythia}  \\  
 157612  &  GGMHiggsinoNLSP\_900\_500\_30\_Zll  &  {\sc Pythia}  \\  
 157613  &  GGMHiggsinoNLSP\_900\_600\_30\_Zll  &  {\sc Pythia}  \\  
 157614  &  GGMHiggsinoNLSP\_900\_700\_30\_Zll  &  {\sc Pythia}  \\  
 157615  &  GGMHiggsinoNLSP\_900\_800\_30\_Zll  &  {\sc Pythia}  \\  
 157616  &  GGMHiggsinoNLSP\_900\_890\_30\_Zll  &  {\sc Pythia}  \\  
 157617  &  GGMHiggsinoNLSP\_1000\_120\_30\_Zll  &  {\sc Pythia}  \\  
 157618  &  GGMHiggsinoNLSP\_1000\_150\_30\_Zll  &  {\sc Pythia}  \\  
 157619  &  GGMHiggsinoNLSP\_1000\_200\_30\_Zll  &  {\sc Pythia}  \\  
 157620  &  GGMHiggsinoNLSP\_1000\_300\_30\_Zll  &  {\sc Pythia}  \\  
 157621  &  GGMHiggsinoNLSP\_1000\_400\_30\_Zll  &  {\sc Pythia}  \\  
 157622  &  GGMHiggsinoNLSP\_1000\_500\_30\_Zll  &  {\sc Pythia}  \\  
 157623  &  GGMHiggsinoNLSP\_1000\_600\_30\_Zll  &  {\sc Pythia}  \\  
 157624  &  GGMHiggsinoNLSP\_1000\_700\_30\_Zll  &  {\sc Pythia}  \\  
 157625  &  GGMHiggsinoNLSP\_1000\_800\_30\_Zll  &  {\sc Pythia}  \\  
 157626  &  GGMHiggsinoNLSP\_1000\_900\_30\_Zll  &  {\sc Pythia}  \\  
 157627  &  GGMHiggsinoNLSP\_1000\_990\_30\_Zll  &  {\sc Pythia}  \\  
\hline
\end{tabular}
\caption{Summary table of the GGM Monte Carlo samples used in the analysis with $\tan{\beta}=30$.}
\label{tab:mcsample_tanbeta30}
%\end{longtable}
\end{center}
\end{table}

%%%%%%%%%%%%%%%%%%%%%%%%%%%%%%%%%%%%%%%%%%%%%%%%%%%%%%%%%%%%%%%%%%%%%%%%%%%%%

%\subsection{Bilinear RPV signal samples}
%\label{sec:bRPV-samples}
%An overview of the phenomenological aspects and determination of model
%parameters has been given in Ref.~\cite{atlas1leplpsupport}.
%Note that previous bRPV signal grids are described in detail 
%in Ref.~\cite{atlas1leplpsupport} and ~\cite{atlasmultijets7TeV}. 
%In summary, bRPV signal points are based on mSUGRA input parameters 
%supplemented by bilinear RPV parameters. The latter 
%are determined by the spectrum generator SPheno~\cite{Porod:2003um, Porod:2011nf} 
%to be consistent with neutrino oscillation results. The mSUGRA parameters in
%this analysis are defined as follows:
%\begin{compactitem} %{itemize}[noitemsep]
%\item $400\GeV \leq m\_0 \leq 2200\GeV$
%\item $250\GeV \leq m\_{1/2} \leq 800\GeV$
%\item $A\_0=-2 m\_0$
%\item $\tan\beta=30$
%\item ${\rm sgn}(\mu)=+1$
%\end{compactitem}
%The main motivation for these mSUGRA parameters is the consistency with a
% Higgs particle of mass around 125~GeV. An overview of Higgs masses 
% for all bRPV signal points is shown in Figure~\ref{fig:bRPV-Higgs-mass}.
% 
% %%%%%%%%%%%%%%%%%%%%%%%%%%%%%%%%%%%%%%%%%%%%%%%%%%%%%%%%%%%%%%%%%%%%%%%%%%%%%%%%%%%
%\begin{figure}[htbp]
%\begin{center}
%\includegraphics[width=0.55\textwidth]{Datasets/fig/bRPV/HiggsAwareGrid\_HiggsMasses.eps}
%%\includegraphics[width=0.84\textwidth]{Datasets/fig/bRPV/higgs\_mass\_2D-117.eps}
%\caption{Higgs mass in $\GeV$ for all bRPV signal points considered.}
%\label{fig:bRPV-Higgs-mass}
%\end{center}
%\end{figure}
%%%%%%%%%%%%%%%%%%%%%%%%%%%%%%%%%%%%%%%%%%%%%%%%%%%%%%%%%%%%%%%%
%
%
%%%%%%%%%%%%%%%%%%%%%%%%%%%%%%%%%%%%%%%%%%%%%%%%%%%%%%%%%%%%%%%%%%%%%%%%% 
%\begin{figure}[hbp]
%\centering
%%\thesubfigure{\includegraphics[width=0.49\linewidth]{Datasets/fig/bRPV/Total\_xsec\_higgsAwareGrid.eps}}\hfill
%%%%\thesubfigure{\includegraphics[width=0.49\linewidth]{Datasets/fig/bRPV/Total\_errors\_higgsAwareGrid.eps}}
%%\thesubfigure{\includegraphics[width=0.49\linewidth]{Datasets/fig/bRPV/Relative\_errors.eps}}
%{\includegraphics[width=0.45\linewidth]{Datasets/fig/bRPV/Total\_xsec\_higgsAwareGrid.eps}}
%\hfill
%{\includegraphics[width=0.5\linewidth]{Datasets/fig/bRPV/Relative\_errors.eps}}
%\caption{Total cross-sections at NLO (NLO+NLL) and relative uncertainties for
%bRPV signal points.}
%\label{fig:bRPV\_total\_xsec\_errors}
%\end{figure} 
% 
%The parameter space for $m\_0$ is extended significantly with
%respect to the previous analysis in Ref.~\cite{atlasmultijets7TeV}, 
%where input masses in the ranges 160\,\GeV\ $\leq m\_0 \leq$ 1400\,\GeV\
%have been taken into account. The bRPV signal grid points have been produced
%in full simulation with 20~k events per point. As in the previous analysis of 
%bRPV signal models, the total cross-sections are mainly dependent on the gaugino
%mass parameter $m\_{1/2}$ setting the scale for the dominant gluino, chargino
%or neutralino production modes. The corresponding total cross-sections and their
%relative uncertainties are shown in Figure~\ref{fig:bRPV\_total\_xsec\_errors}.
%An overview of bRPV SUSY
%production processes is given in Appendix~\ref{sec:SUSY\_prod} and corresponding
%decays are displayed in Appendix~\ref{sec:bRPV\_decays}.

%%%%%%%%%%%%%%%%%%%%%%%%%%%%%%%%%%%%%%%%%%%%%%%%%%%%%%%%%%%%%%%%%%%%%%%%%%%%%

%%%%%%%%%%%%%%%%%%%%%%%%%%%%%%%%%%%%%%%%%%%%%%%%%%%%%%%%%%%%%%%%%%%%%%%%%%%%%

%\subsection{Simplified two-step $WWZZ$ signal samples}
%\label{sec:WWZZ-samples}
%This model and the signal samples generated is described in~\cite{COMB\_INT}.


